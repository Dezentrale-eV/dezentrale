\documentclass[10pt,a4paper]{scrartcl}
\usepackage{eurosym}
\usepackage[utf8x]{inputenc}
\usepackage{ngerman}
\usepackage[ngerman]{babel}
\usepackage[margin=3cm]{geometry}
\newcommand{\qt}[1]{\glq\emph{#1}\grq}
\newcommand{\qs}[1]{\glqq#1\grqq}
\newcommand{\name}{dezentrale}
\newcommand{\revision}{$Revision: 2017-07-02$}
\newcommand{\eventdate}{02.07.2017}
\newcommand{\schriftfuehrer}{Jan Hollburg}
\newcommand{\documentstatus}{Ver{\"o}ffentlicht (PUBLIC)}
\setlength{\parskip}{6pt}
\setlength{\parindent}{0pt}
\usepackage{enumerate}
\usepackage{color}
\pagestyle{plain}
\usepackage{palatino}
\usepackage[bookmarks,bookmarksopen=true,bookmarksnumbered=true,colorlinks,linkcolor=black,urlcolor=blue]{hyperref}
\begin{document}
\title{Protokoll - Vorstandssitzung \qs{\name\ e.V.}}
{\LARGE Protokoll - Vorstandssitzung \qs{\name\ e.V.}}

\section*{Dokumentenstatus}
\documentstatus\\
Fassung vom \eventdate\ (\revision)

\section*{1) {\"U}berblick}
    Ort: Sublab e.V., Karl-Heine-Str. 93, Leipzig\\
    Datum: \eventdate\\
    Anzahl der anwesenden Vorstandsmitglieder: 3\\
    Anzahl der anwesenden Beisitzer: 1\\
    Schriftf{\"u}hrer: \schriftfuehrer
\subsection*{Beschlussf{\"a}higkeit}
    Es sind 7 von 9 Stimmgewichte anwesend. Der Vorstand ist beschlussf{\"a}hig.

\section*{2) Tagesordnung}
    Die folgenden Punkte stehen auf der Tagesordnung der Sitzung:
    \begin{itemize}
        \item {\"U}berarbeitung der Satzung des Vereins nach Vorgaben des Finanzamtes
        \item Abstimmung {\"u}ber den {\"A}nderungsbeschluss
        \item Festlegung der weiteren Schritte
    \end{itemize}

\section*{3) {\"U}berarbeitung der Satzung des Vereins nach Vorgaben des Finanzamtes}
    Die Vorpr{\"u}fung des Finanzamtes hat ergeben, dass die bisherige Formulierung zur
    Gemeinn{\"u}tzigkeit nach {\"A}nderung durch das Jahressteuergesetz 2009 nicht mehr ausreichend ist.
    Folgender Brief erreichte uns:
\subsection*{Eingehender Brief (Auszug)}
    \begin{itemize}
        \item \qs{	Durch das Jahressteuergesetz 2009 tritt eine {\"A}nderung der formellen
					Voraussetzungen f{\"u}r die Steuerbeg{\"u}nstigung in Kraft. Die bisher
					unverbindliche Mustersatzung ist nunmehr festgeschrieben.}
		\item \qs{	Gem. \S\ 59 AO ist Voraussetzung f{\"u}r die Anerkennung der Gemeinn{\"u}tzigkeit,
					dass sich aus der Satzung ergibt, welchen Zweck die K{\"o}rperschaft
					verfolgt, dass dieser Zweck den Anforderungen der \S\S\ 52 bis 55
					entspricht und dass er ausschlie{\ss}lich und unmittelbar verfolgt wird; die
					tats{\"a}chliche Gesch{\"a}ftsf{\"u}hrung muss diesen Satzungsbestimmungen
					entsprechen.}
		\item \qs{	Das bedeutet, das aus der Satzung u. a. der steuerbeg{\"u}nstige Zweck der
					\S\S\ 52 bis 54 AO (z. B. Umweltschutz) direkt hervorgeht. Au{\ss}erdem ist die
					pr{\"a}zise, nachpr{\"u}fbare Angabe des gemeinn{\"u}tzigen Zwecks und die jeweilige
					Art der Verwirklichung erforderlich.}
		\item \qs{	Die Darstellung, wie der Verein seine Ziele verwirklichen will, muss so
					konkret sein, dass ein Interessierter genau erkennen und beurteilen
					kann, mit was sich der Verein besch{\"a}ftigt.}
		\item \qs{	Aus der Satzung ist nicht zu erkennen, wie die internationale Gesinnung
					und V{\"o}lkerverst{\"a}ndigung gef{\"o}rdert werden soll.
					Unter der F{\"o}rderung der internationalen Gesinnung und
					V{\"o}lkerverst{\"a}ndigung ist die Entwicklung und St{\"a}rkung freundschaftlicher
					Beziehungen zwischen den V{\"o}lkern zu verstehen. Diesem Ziel dienen alle
					Aktivit{\"a}ten, die zur zwischenmenschlichen Begegnung der Angeh{\"o}rigen
					verschiedener V{\"o}lker beitragen, das Wissen {\"u}ber andere V{\"o}lker mehren und
					die Einsicht in die Vorteile friedlichen Zusammenlebens f{\"o}rdern.}
		\item \qs{	Die geplanten Ma{\ss}nahmen zur F{\"o}rderung der demokratischen Staatswesens
					f{\"u}hren nicht zur F{\"o}rderung desselben.}
		\item \qs{	Eine F{\"o}rderung des demokratischen Staatswesens liegt vor, wenn sich in
					K{\"o}rperschaft umfassend mit den demokratischen Grundprinzipien befasst
					und diese objektiv und neutral w{\"u}rdigt. Vortr{\"a}ge und Diskussionen zu
					Themen wie Urheberrecht, Datenschutz u. {\"a}
					geh{\"o}ren nicht dazu. Hierbei handelt es sich um gesetsliche Regelungen.}
		\item \qs{	Ich empfehle die Satzung entsprechend zu {\"u}berarbeiten.}
	\end{itemize}

\subsection*{Satzung \S\ 2 Punkt 2 wird ge{\"a}ndert. Alter \S\ 2.2:}
	Insbesondere in (jedoch nicht begrenzt auf) dem Rahmen der folgenden Mittel:
	\begin{itemize}
		\item Aufbau einer Begegnungsst{\"a}tte f{\"u}r Veranstaltungen, Experimente, kommunikativem Austausch, etc.
		\item Regelm{\"a}{\ss}ige {\"o}ffentliche Treffen und Informationsveranstaltungen
		\item Veranstaltungen und/oder F{\"o}rderung internationaler Kongresse, Treffen
		\item {\"O}ffentlichkeitsarbeit und Telepublishing in allen Medien
		\item F{\"o}rderung des sch{\"o}pferisch-kritischen Umgangs mit Technologie
		\item Bildung und Weiterbildung zu technischen Fragen
	\end{itemize}
	setzt sich der Verein ein f{\"u}r die F{\"o}rderung von:
	\begin{itemize}
		\item Erziehung, Volksbildung und Studentenhilfe (AO \S \ 52 2.7) in den in der Pr{\"a}ambel angesprochenen Themen,
			insbesondere der digitalen Informationsverarbeitung und deren Einfluss auf die Gesellschaft;
			durchgef{\"u}hrt durch Bildungsveranstaltungen, Experimentierr{\"a}ume und -projekte sowie Informationsaustausch.
		\item Kunst und Kultur (AO \S \ 52 2.5.) in bestehenden und neuen Formen, wie sie durch Einfl{\"u}sse der digitalen
			Informationsverarbeitung entstanden sind und entstehen, z.\, B. NetArt, BlinkenLights und andere Computerkunst.
		\item Internationaler Gesinnung und V{\"o}lkerverst{\"a}ndigung (AO \S \ 52 2.13.) durch Austausch mit {\"a}hnlichen oder gleichgesinnten
			Vereinen, Einrichtungen und Projekten.
		\item Kriminalpr{\"a}vention (AO \S \ 52 2.20.) insbesondere im Umgang mit digitaler Informationsverarbeitungstechnik durch
			Aufkl{\"a}rung {\"u}ber rechtliche Grunds{\"a}tze, angemessene Verhaltensweisen und Unterbreitung von Alternativen zu 
			kriminellen Handlungsweisen
		\item Demokratischem Staatswesen (AO \S \ 52 2.24.) im Besonderen im Zusammenhang mit der Entwicklung der Gesellschaft 
			zu einer Informationsgesellschaft durch Veranstaltungen und Diskussionen zu Themen wie Urheberrecht, Datenschutz, 
			Netzneutralit{\"a}t, freie und offene Software, etc.
	\end{itemize}

\subsection*{Neuer \S\ 2.2:}
	Der Satzungszweck wird verwirklicht insbesondere durch folgende Ma{\ss}nahmen:
	\begin{itemize}
		\item	Kunst und Kultur (AO \S\ 52 2.5) durch die Ausbildung an und die Nutzung von technischen Ger{\"a}ten zur
				Selbstentfaltung und Schaffung von Kunst. Hierf{\"u}r gibt es regelm{\"a}{\ss}ige {\"o}ffentliche Workshops und Treffen
				(zum Beispiel die \qs{Hardware-Bastelrunde}), unregelm{\"a}{\"ss}ige {\"o}ffentliche Veranstaltungen (zum Beispiel
				"Hebocon"), Pr{\"a}sentationsm{\"o}glichkeit f{\"u}r K{\"u}nstler und Projekte (zum Beispiel \qs{Projektionsmapping}
				in den Vereinsr{\"a}umen) und {\"o}ffentlich sichtbare Medien f{\"u}r Kunst in der Informationstechnologie
				(zum Beispiel interaktive Installationen im Schaufenster).
		\item	Erziehung und Bildung (AO \S\ 52 2.7) durch regelm{\"a}{\ss}ige {\"o}ffentliche Workshops und Treffen (zum Beispiel
				Sicherheit in der Informationstechnologie \qs{Cyber-Sec Gruppe}, einsteigerfreundliche Programmiertreffen)
				und offene Selbsthilfetreffen zur Erkundung der inneren Funktionsweise, Fehleranalyse und Reparatur von
				defekten Ger{\"a}ten (\qs{Technik-Sprechstunde}).
		\item	Kriminalpr{\"a}vention (AO \S\ 52 2.20) durch allgemeine und individuelle Ratschl{\"a}ge zur Sicherung der eigenen
				Daten und des Computers, Hilfe bei kompromittierten Daten sowie weitere M{\"o}glichkeiten der
				opferbasierten Pr{\"a}vention von digitaler Kriminalit{\"a}t.\\
				Hierzu gibt es fortlaufend Ansprechpartner in den Vereinsr{\"a}umen und in den digitalen Kommunikationskan{\"a}len,
				regelm{\"a}{\ss}ige {\"o}ffentliche Workshops und Treffen (zum Beispiel die \qs{Cyber-Sec Gruppe}), unregelm{\"a}{\ss}ige
				{\"o}ffentliche Veranstaltungen (zum Beispiel j{\"a}hriche Konferenzen) und die M{\"o}glichkeit externer Gruppen
				(zum Beispiel der B{\"u}ndnis Privatsph{\"a}re Leipzig e.V.) {\"o}ffentliche Veranstaltungen zu diesem Themenkomplex
				in den Vereinsr{\"a}umen durchzuf{\"u}hren.
	\end{itemize}
\section*{4) Abstimmung {\"u}ber den {\"A}nderungsbeschluss}
    \begin{enumerate}
        \item Die {\"A}nderung von \S\ 2 Punkt 2 wird einstimmig von der Vorstandssitzung beschlossen.\\
    \end{enumerate}
\section*{5) Festlegung der weiteren Schritte}
    Die weiteren Schritte werden wie folgt abgearbeitet:
    \begin{itemize}
        \item Erneute Beglaubigung / Vorbereitung der Anmeldung beim Notar
		\item {\"U}bermittlung an das Finanzamt zur erneuten Vorpr{\"u}fung; Frage ans Finanzamt: Welche Nachweise sind n{\"o}tig und gew{\"u}nscht?
		\item Einreichung beim Amtsgericht
    \end{itemize}

\section*{6) Ende der Sitzung}
    Mit ihren Unterschriften best{\"a}tigt der Schriftf{\"u}hrer die Inhalte dieses Protokolls.
\\
\\
\\
\\
Leipzig, den \eventdate \ \ \ Schriftf{\"u}hrer: \schriftfuehrer
\end{document}

