\documentclass[10pt,a4paper]{scrartcl}
\usepackage{eurosym}
\usepackage[margin=3cm]{geometry}
\usepackage[utf8]{inputenc}
\newcommand{\qt}[1]{\glq\emph{#1}\grq}
\newcommand{\qs}[1]{"#1"}
\newcommand{\name}{dezentrale}
\newcommand{\revision}{$4$}
\newcommand{\revdate}{$2018-04-23$}
\setlength{\parskip}{6pt}
\setlength{\parindent}{0pt}
\usepackage{enumerate}
\usepackage{color}
\usepackage{svg}
\pagestyle{plain}
\usepackage{palatino}
\usepackage[bookmarks,bookmarksopen=true,bookmarksnumbered=true,colorlinks,linkcolor=black,urlcolor=blue]{hyperref}
\begin{document}

{\setlength\tabcolsep{0 pt}
\begin{tabular}{p{10.6cm}p{4.7cm}} {\LARGE Satzung des \name\ e.V.} &
	\begin{tabular}{@{}c@{}}
	\includesvg[width=4.5cm]{dz_logo_vektor_alpha}
	\end{tabular}
\end{tabular}
}
Fassung vom \revdate\ (Revision \revision)

\subsection*{Pr{\"a}ambel}



Wir sind eine Gemeinschaft von Personen, unabh{\"a}ngig von Alter,
Geschlecht, Herkunft, politischer Ausrichtung und gesellschaftlicher
Stellung, die sich im Umgang mit Informationstechnologie bildet, damit
verwandte und nahestehende Formen von Kunst und Kultur f{\"o}rdert und sich f{\"u}r
eine jedem Wesen gerechte Entwicklung der Informationsgesellschaft
einsetzt.
%
% Paragraph 1 ====================================================
%
\subsection*{\S \ 1 Name, Sitz, Gesch{\"a}ftsjahr}
\begin{enumerate}
\item Der Verein tr{\"a}gt den Namen \qs{\name}. Der Verein wird in das
Vereinsregister beim Amtsgericht Leipzig eingetragen
und der Name dann um den Zusatz \qs{e.V.} erg{\"a}nzt.
\item Der Verein hat seinen Sitz in Leipzig. Das Gesch{\"a}ftsjahr entspricht
dem Kalenderjahr.
\end{enumerate}
%
% Paragraph 2 ====================================================
%
\subsection*{\S \ 2 Zweck, Gemeinn{\"u}tzigkeit}
\begin{enumerate}
\item Der Verein verfolgt ausschlie{\ss}lich und unmittelbar gemeinn{\"u}tzige
Zwecke im Sinne des Abschnitts \qs{steuerbeg{\"u}nstigte Zwecke} der
Abgabenordnung 1977 (\S \ 51 ff AO) in der jeweils g{\"u}ltigen Fassung;
er dient ausschlie{\ss}lich und unmittelbar den in Absatz 2 angegebenen
Zwecken. Er ist selbstlos t{\"a}tig und verfolgt nicht in erster Linie eigenwirtschaftliche Zwecke.
Die Mittel des Vereins werden ausschlie{\ss}lich und unmittelbar zu den
satzungsgem{\"a}{\ss}en Zwecken verwendet. Die Mitglieder erhalten keine
Zuwendung aus den Mitteln des Vereins. Die Mitglieder d{\"u}rfen bei
ihrem Ausscheiden oder bei Aufl{\"o}sung des Vereins keine Anteile des
Vereinsverm{\"o}gens erhalten. Niemand darf durch Ausgaben, die dem
Zwecke des Vereins fremd sind, oder durch unverh{\"a}ltnism{\"a}{\ss}ig hohe
Ver\-g{\"u}\-tung\-en be\-g{\"u}ns\-tigt werden.

\item 	Der Satzungszweck wird verwirklicht insbesondere durch folgende Ma{\ss}nahmen:
	\begin{itemize}
		\item	Kunst und Kultur (AO \S\ 52 2.5) durch die Ausbildung an und die Nutzung von technischen Ger{\"a}ten zur
				Selbstentfaltung und Schaffung von Kunst. Hierf{\"u}r gibt es re\-gel\-m{\"a}{\ss}ige {\"o}ffent\-liche Workshops und Treffen
				(zum Beispiel die \qs{Hard\-ware-Bastel\-runde}), unregelm{\"a}{\"ss}ige {\"o}ffent\-liche Veran\-stal\-tungen (zum Beispiel
				"Hebocon"), Pr{\"a}\-senta\-tions\-m{\"o}glichkeit f{\"u}r K{\"u}nstler und Projekte (zum Beispiel \qs{Projektionsmapping}
				in den Vereins\-r{\"a}umen) und {\"o}ffent\-lich sichtbare Medien f{\"u}r Kunst in der Infor\-mations\-techno\-logie
				(zum Beispiel interaktive Installationen im Schaufenster).
		\item	Erziehung und Bildung (AO \S\ 52 2.7) durch regel\-m{\"a}{\ss}ige {\"o}ffent\-liche Workshops und Treffen (zum Beispiel
				Sicherheit in der Informationstechnologie \qs{Cyber-Sec Gruppe}, einsteigerfreundliche Programmiertreffen)
				und offene Selbsthilfetreffen zur Erkun\-dung der inneren Funktionsweise, Fehleranalyse und Reparatur von
				defekten Ge\-r{\"a}ten (\qs{Tech\-nik-Sprech\-stunde}).
		\item	Kriminalpr{\"a}vention (AO \S\ 52 2.20) durch allgemeine und individuelle Ratschl{\"a}ge zur Sicherung der eigenen
				Daten und des Computers, Hilfe bei kompromittierten Daten sowie weitere M{\"o}glichkeiten der
				opferbasierten Pr{\"a}vention von digitaler Kriminalit{\"a}t.\\
				Hierzu gibt es fortlaufend Ansprechpartner in den Vereinsr{\"a}umen und in den digitalen Kommunikationskan{\"a}len,
				regelm{\"a}{\ss}ige {\"o}ffentliche Workshops und Treffen (zum Beispiel die \qs{Cyber-Sec Gruppe}), unregelm{\"a}{\ss}ige
				{\"o}ffentliche Veranstaltungen (zum Beispiel j{\"a}hriche Konferenzen) und die M{\"o}glichkeit externer Gruppen
				(zum Beispiel der B{\"u}ndnis Privatsph{\"a}re Leipzig e.V.) {\"o}ffentliche Veranstaltungen zu diesem Themenkomplex
				in den Vereinsr{\"a}umen durchzuf{\"u}hren.\\
	\end{itemize}
\end{enumerate}
%
% Paragraph 3 ====================================================
%
% [1]
\subsection*{\S \ 3 Mitgliedschaft}
\begin{enumerate}
	\item Mitglied des Vereins kann jede nat{\"u}rliche oder juristische Person oder
		nicht rechts\-f{\"a}higer Verein werden, die seine Ziele unterst{\"u}tzt.
        \item {\"U}ber die Aufnahme von Mitgliedern entscheidet der Vorstand. Eine Ablehnung muss zeitnah schriftlich gegen{\"u}ber dem Antragsteller begr{\"u}ndet werden.
	\item Die Beitrittserkl{\"a}rung erfolgt schriftlich gem{\"a}{\ss} \S \ 12 gegen{\"u}ber dem Vorstand. Die
		Mitgliedschaft beginnt mit der Aush{\"a}ndigung einer entsprechenden Best{\"a}tigung durch
                ein Vorstandsmitglied und Zahlung des ersten Mitgliedsbeitrages.
	\item Hat der Vorstand die Aufnahme abgelehnt, so kann der Mitgliedschaftsbewerber Einspruch
		zur n{\"a}chsten Mitgliederversammlung einlegen, die dann abschlie{\ss}end {\"u}ber die Aufnahme
		oder Nichtaufnahme entscheidet.
	\item Die Mitgliedschaft endet durch Austrittserkl{\"a}rung, durch Ausschluss, durch Tod von
		nat{\"u}r\-li\-chen Personen oder durch Aufl{\"o}sung und Erl{\"o}schung von nicht
		nat{\"u}r\-lichen Personen.
		Die Beitragspflicht f{\"u}r das laufende Gesch{\"a}ftsjahr wird von der Gesch{\"a}ftsordnung
		geregelt.
	\item Der Austritt wird durch schriftliche Willenserkl{\"a}rung gem{\"a}{\ss} \S \ 12 gegen{\"u}ber dem Vorstand
		erkl{\"a}rt.
\end{enumerate}
%
% Paragraph 4 ====================================================
%
\subsection*{\S \ 4 Ausschluss eines Mitglieds }
\begin{enumerate}
	\item Ein Mitglied kann durch Beschluss des Vorstandes ausgeschlossen werden, wenn es das
                Ansehen des Vereins sch{\"a}digt, sich innerhalb des Vereines wiederholt respektlos oder 
                beleidigend {\"a}ussert, sich gruppenbezogen menschenfeindlich bet{\"a}tigt, seinen Beitragsverpflichtungen nicht nachkommt oder 
		wenn ein sonstiger wichtiger Grund vorliegt. Der Vorstand muss dem auszuschlie{\ss}enden
		Mitglied den Beschluss in schriftlicher Form gem{\"a}{\ss} \S \ 12 unter Angabe von Gr{\"u}nden
		mitteilen und ihm auf Verlangen eine Anh{\"o}rung gew{\"a}hren.
        \item Gegen diesen Beschluss des Vorstandes ist die Anrufung der Mitgliederversammlung zu\-l{\"a}s\-sig. Die Mitgliederversammlung kann dann den Vorstandsbeschluss {\"u}berstimmen.
				Bis zur Entscheidung der Mitgliederversammlung ruht die Mitgliedschaft.
		\item Zur {\"U}berstimmung des Vorstandsbeschlusses zum Ausschluss eines Mitgliedes ist eine Mehrheit von zwei Dritteln in der Mitgliederversammlung erforderlich.
	
\end{enumerate}
%
% Paragraph 5 ====================================================
%
\subsection*{\S \ 5 Rechte und Pflichten der Mitglieder}
\begin{enumerate}
	\item Die Mitglieder sind berechtigt, die Leistungen des Vereins entsprechend der vorhandenen
		M{\"o}glichkeiten und in angemessenem und verh{\"a}ltnism{\"a}{\ss}igem Ausma{\ss} in Anspruch zu nehmen.
                N{\"a}\-he\-res kann die Gesch{\"a}ftsordnung regeln.
	\item Die Mitglieder sind verpflichtet, die satzungsgem{\"a}{\ss}en Zwecke des Vereins zu unterst{\"u}tzen
		und zu f{\"o}rdern.
	\item Der Verein erhebt einen Mitgliedsbeitrag, zu dessen Zahlung die Mitglieder verpflichtet
		sind. N{\"a}heres regelt die Gesch{\"a}ftsordnung.
        
\end{enumerate}
%
% Paragraph 6 ====================================================
%
\subsection*{\S \ 6: Hausrecht}
\begin{enumerate}
	\item Der Vorstand übertr{\"a}gt allen Mitgliedern ein
		eingeschr{\"a}nktes Hausrecht.
	\item Mitglieder mit eingeschr{\"a}nktem Hausrecht sind
		befugt, nach eigenem Ermessen Personen der
		R{\"a}umlichkeiten zu verweisen, um damit materiellen
		oder immateriellen Schaden abzuwenden oder zu
		minimieren. Ein solcher Verweis ist auf maximal 24
		Stunden begrenzt. Das Mitglied muss unmittelbar
		den Vorstand {\"u}ber die Situation in Kenntnis setzen.
	\item Der Vorstand kann ein weitergehendes Hausrecht
		{\"u}bertragen und auch wieder entziehen.
\end{enumerate}
%
% Paragraph 7 ====================================================
%
\subsection*{\S \ 7 Organe des Vereins }
\begin{enumerate}
	\item Die Organe des Vereins sind:
		\begin{itemize}
			\item Die Mitgliederversammlung
			\item Der Vorstand
		\end{itemize}
\end{enumerate}
%
% Paragraph 8 ====================================================
%
\subsection*{\S \ 8 Mitgliederversammlung}
\begin{enumerate}

        \item Die Mitgliederversammlung ist das oberste Beschlussorgan des Vereins.
	\item Beschl{\"u}sse werden von der Mitgliederversammlung durch {\"o}ffentliche Abstimmung getroffen.
		Auf Wunsch eines Mitglieds ist geheim abzustimmen.
	\item Jedes Mitglied hat genau eine Stimme.
	\item Zur Fassung eines Beschlusses ist eine einfache Mehrheit der abgegebenen Stimmen
             not\-wen\-dig, sofern keine abweichende Regelung in Satzung oder Gesch{\"a}ftsordnung getroffen wurde.
			 Eine zur Herstellung der Beschlussf{\"a}higkeit n{\"o}tige Grenze von abgegebenen Stimmen wird in der
			 Gesch{\"a}ftsordnung festgelegt.
	\item Eine ordentliche Mitgliederversammlung, bezeichnet als Jahreshauptversammlung,
		wird einmal j{\"a}hrlich einberufen. Ihre Tagesordnung umfasst unter anderem die
		Vorstellung des Rechenschaftsberichts f{\"u}r das vorherige Gesch{\"a}ftsjahr durch
		den Schatzmeister.
	\item Eine au{\ss}erordentliche Mitgliederversammlung kann jederzeit einberufen werden, wenn
		mindestens 23\% der Mitglieder oder der Vorstand dies jeweils schriftlich gem{\"a}{\ss} \S \ 12
		unter Angabe eines Grunds beantragen. Dem angegebenen Grund m{\"u}ssen die gew{\"u}nschten
		Tagesordnungspunkte zu entnehmen sein; sie werden auf die Einladung {\"u}bernommen.
	\item Dem Vorstand obliegt zu allen Mitgliederversammlungen die Festsetzung eines Termins
		und die rechtzeitige Einladung aller Mitglieder bis sp{\"a}testens 2 Wochen vor dem
		von ihm festgesetzten Termin. Bei von den Mitgliedern beantragten
                Mitgliederversammlungen darf der Termin nicht mehr als 4 Wochen nach dem Eingang
		des Antrags beim Vorstand liegen.
	\item Der Vorstand stellt die Einladungen auf schriftlichem Weg gem{\"a}{\ss} \S \ 12 zu, muss
		jedoch eine Kopie auf dem Postweg zustellen, falls das Mitglied den Wunsch dazu
		schriftlich gem{\"a}{\ss} \S \ 12 angemeldet hat.
	\item In der Einladung werden die Tagesordnungspunkte sowie weitere n{\"o}tige Informationen
		bekannt gegeben. Die Mitgliederversammlung kann per Beschluss die Tagesordnung
		ver\-{\"a}ndern.
	\item {\"U}ber die Beschl{\"u}sse der Mitgliederversammlung ist ein Protokoll anzufertigen,
		das vom Versammlungsleiter und dem Schriftf{\"u}hrer zu unterzeichnen ist.
		Das Protokoll ist innerhalb von 14 Tagen allen Mitgliedern zug{\"a}nglich zu
		machen und auf der n{\"a}chsten Mitgliederversammlung genehmigen zu lassen.
	\item Der Vorstandsvorsitzende ist Versammlungsleiter der Mitgliederversammlung.
		Die Mitgliederversammlung kann durch Beschluss einen anderen Versammlungsleiter
		oder Schrift\-f{\"u}h\-rer bestimmen.
\end{enumerate}
%
% Paragraph 9 ====================================================
%
\subsection*{\S \ 9 Vorstand }
\begin{enumerate}

    \item Der Vorstand besteht aus mindestens drei Mitgliedern: dem
    Vorstandsvorsitzenden, dem Schatzmeister und dem Schriftf{\"u}hrer. Des
    Weiteren k{\"o}nnen bis zu drei Beisitzer in den Vorstand gew{\"a}hlt werden. Es
    kann auf Wunsch der Mitglieder auf eine Wahl der Beisitzer verzichtet
    werden. Die Gesch{\"a}ftsordnung regelt das verwendete Wahlverfahren.
    \item Vorstand im Sinne des \S \ 26 BGB ist jedes Vorstandsmitglied.
    Vor\-stands\-vorsitzen\-der, Schatz\-meis\-ter und Schriftf{\"u}hrer sind einzeln
    berechtigt, den Verein nach au{\ss}en zu vertreten. Die Gesch{\"a}ftsordnung kann
    hierf{\"u}r Einschr{\"a}nkungen festlegen.
	\item Der Schatzmeister {\"u}ber\-wacht die Haushaltsf{\"u}hrung und verwaltet das
		Ver\-m{\"o}\-gen des Vereins. N{\"a}\-her\-es regelt die Ge\-sch{\"a}fts\-ord\-nung.
	\item Vorstandsmitglieder k{\"o}nnen jederzeit von ihrem Amt zur{\"u}cktreten.
        \item F{\"u}r die Abberufung des Vorstandes ist eine Mehrheit von zwei Dritteln in der Mitgliederversammlung erforderlich. Der Vorstand kann nur als Ganzes abberufen werden und die Abstimmung muss in der Einladung zur Mitgliederversammlung als Tagesordnungspunkt aufgef{\"u}hrt werden.
        \item Bei R{\"u}cktritt oder andauernder Aus{\"u}bungsunf{\"a}higkeit eines Vorstandsmitgliedes oder Abberufung
		des Vorstandes ist der gesamte Vorstand neu zu w{\"a}hlen. Bis zur Wahl eines neuen Vorstands ist der
		bisherige Vorstand zur bestm{\"o}glichen Wahrnehmung seiner Aufgaben verpflichtet.
    \item Die Amtsdauer der Vorstandsmitglieder betr{\"a}gt zwei Jahre. Sie werden
    von der Mit\-glieder\-ver\-sammlung aus den Mitgliedern des Vereins gew{\"a}hlt. Es
    werden nacheinander Schatz\-meister, Vorstands\-vorsitzen\-der und Schrift\-f{\"u}hrer
    sowie falls gew{\"u}nscht bis zu drei Beisitzer gew{\"a}hlt. Die Wiederwahl ist
    zul{\"a}ssig.
    \item Dem Vorstand obliegen alle Entschei\-dungen, die nicht durch die Satzung oder die Ge\-sch{\"a}fts\-ord\-nung geregelt sind.
	\item Der Vorstand ist Dienstvorgesetzter aller vom Verein angestellten Mitarbeiter;
		er kann diese Aufgabe einem Vorstandsmitglied {\"u}bertragen.
	\item Die Vorstandsmitglieder sind grunds{\"a}tzlich ehrenamtlich t{\"a}tig; sie haben Anspruch
		auf Erstattung notwendiger Auslagen, deren Rahmen von der Gesch{\"a}ftsordnung
		festgelegt wird.
    \item Der Vorstand tritt nach Bedarf zusammen. Die Vorstands\-sitzungen
    werden vom Schrift\-f{\"u}h\-rer schriftlich einberufen. Der Vorstand ist
    beschluss\-f{\"a}hig wenn mindestens zwei Drittel (nach Stimmgewicht) der Vor\-stands\-mit\-glieder
    teilnehmen. Die Beschl{\"u}sse der Vor\-stands\-sitzung sind schriftlich zu
    protokollieren. Protokolle sind auf Anfrage Vereinsmitgliedern zu\-g{\"a}ng\-lich zu machen.
    \item Vorstandsvorsitzender, Schatzmeister und Schriftf{\"u}hrer haben bei
    Abstimmungen des Vorstands jeweils zwei Stimmen. Jeder Beisitzer hat eine
    Stimme. Bei Abstimmungen ist eine Mehrheit von zwei Dritteln der
    abgegebenen g{\"u}ltigen Stimmen n{\"o}tig.
\end{enumerate}
%
% Paragraph 10 ====================================================
%
\subsection*{\S \ 10 Satzungs- und Gesch{\"a}ftsordnungs{\"a}nderung}
\begin{enumerate}
	\item {\"U}ber Satzungs- und Gesch{\"a}ftsordnungs{\"a}nderungen kann in der Mitgliederversammlung
		nur abgestimmt werden, wenn auf diesen Tagesordnungspunkt hingewiesen wurde und der
		Einladung sowohl der bisherige als auch der vorgesehene neue Text beigef{\"u}gt
		worden war.
	\item F{\"u}r die Satzungs- oder Gesch{\"a}ftsordnungs{\"a}nderung ist eine Mehrheit von zwei
		Dritteln in der Mitgliederversammlung erforderlich.
	\item Satzungs{\"a}nderungen, die von Aufsichts-, Gerichts- oder Finanz\-be\-h{\"o}r\-den aus formalen
		Gr{\"u}n\-den verlangt werden, kann der Vorstand von sich aus vornehmen. Diese
		Sat\-zungs\-{\"a}n\-der\-ung\-en m{\"u}s\-sen der n{\"a}chs\-ten Mitgliederversammlung mitgeteilt
		werden.
\end{enumerate}
%
% Paragraph 11 ====================================================
%
\subsection*{\S \ 11 Aufl{\"o}sung des Vereins und Verm{\"o}gensbindung}
\begin{enumerate}
	\item Die Aufl{\"o}sung des Vereins muss von der Mitgliederversammlung mit einer Mehrheit von
		zwei Dritteln beschlossen werden. Die Abstimmung ist nur m{\"o}glich, wenn auf der Einladung
		zur Mitgliederversammlung als einziger Tagesordnungspunkt die Aufl{\"o}sung des Vereins
		angek{\"u}ndigt wurde.
	\item Bei Aufl{\"o}sung des Vereins, Aufhebung der Rechtsf{\"a}higkeit oder Wegfall der
		gemeinn{\"u}tzigen Zwecke darf das Verm{\"o}gen der K{\"o}rperschaft nur f{\"u}r
		steuerbeg{\"u}nstigte Zwecke verwendet werden. Zur Erf{\"u}llung dieser
		Voraussetzung wird das Verm{\"o}gen einer anderen steu\-er\-be\-g{\"u}ns\-tig\-ten
		K{\"o}rperschaft oder einer K{\"o}rperschaft {\"o}ffentlichen Rechts f{\"u}r
		steuerbeg{\"u}nstigte Zwecke {\"u}ber\-tra\-gen, die ebenfalls den Auftrag
		zur Bildung und Volksbildung im Umgang mit Informationstechnologie
		wahrnimmt. N{\"a}\-he\-res kann die Gesch{\"a}ftsordnung regeln.
	\item Der Grundsatz der Verm{\"o}gensbindung ist bei der Fassung von
		Beschl{\"u}ssen {\"u}ber die k{\"u}nf\-ti\-ge Verwendung des Vereinsverm{\"o}gens zwingend
		zu erf{\"u}llen.
	\item Bei Verlust der Anerkennung als gemeinn{\"u}tziger Verein gelten die vorgenannten Abs{\"a}tze analog, 
		das Verm{\"o}gen und die G{\"u}ter des Vereins werden entsprechend {\"u}bertragen.
\end{enumerate}
%
% Paragraph 12 ====================================================
%
\subsection*{\S \ 12 Schriftliche Kommunikation, Abstimmungsf{\"a}higkeit}
\begin{enumerate}
	\item Schriftliche Erkl{\"a}rungen im Sinne dieser Satzung sind
		handschriftlich unterschriebene Dokumente in Papierform sowie
		mit PGP oder S/MIME signierte elektronische Dokumente.
	\item Zu Mitgliederversammlungen werden elektronisch nach Abs. 1,
		fernm{\"u}ndlich oder postalisch zugestellte Stimmen von Mitgliedern
		wie Stimmen von anwesenden Mitgliedern gez{\"a}hlt.
\end{enumerate}
\end{document}

