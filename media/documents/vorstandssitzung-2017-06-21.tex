\documentclass[10pt,a4paper]{scrartcl}
\usepackage{eurosym}
\usepackage[utf8x]{inputenc}
\usepackage{ngerman}
\usepackage[ngerman]{babel}
\usepackage[margin=3cm]{geometry}
\newcommand{\qt}[1]{\glq\emph{#1}\grq}
\newcommand{\qs}[1]{\glqq#1\grqq}
\newcommand{\name}{dezentrale}
\newcommand{\revision}{$Revision: 2017-06-21$}
\newcommand{\eventdate}{21.06.2017}
\newcommand{\schriftfuehrer}{Jan Hollburg}
\newcommand{\documentstatus}{Ver{\"o}ffentlicht (PUBLIC)}
\setlength{\parskip}{6pt}
\setlength{\parindent}{0pt}
\usepackage{enumerate}
\usepackage{color}
\pagestyle{plain}
\usepackage{palatino}
\usepackage[bookmarks,bookmarksopen=true,bookmarksnumbered=true,colorlinks,linkcolor=black,urlcolor=blue]{hyperref}
\begin{document}
\title{Protokoll - Vorstandssitzung \qs{\name\ e.V.}}
{\LARGE Protokoll - Vorstandssitzung \qs{\name\ e.V.}}

\section*{Dokumentenstatus}
\documentstatus\\
Fassung vom \eventdate\ (\revision)

\section*{1) {\"U}berblick}
    Ort: Sublab e.V., Karl-Heine-Str. 93, Leipzig\\
    Datum: \eventdate\\
	Anzahl der anwesenden Vorstandsmitglieder: 3\\
	Anzahl der anwesenden Beisitzer: 2\\
    Schriftf{\"u}hrer: \schriftfuehrer
\subsection*{Beschlussf{\"a}higkeit}
    Es sind 8 von 9 Stimmgewichte anwesend. Der Vorstand ist beschlussf{\"a}hig.

\section*{2) Tagesordnung}
    Die folgenden Punkte stehen auf der Tagesordnung der Sitzung:
	\begin{itemize}
        \item {\"U}berarbeitung der Satzung des Vereins nach Vorgaben des Amtsgerichts
		\item Abstimmung {\"u}ber den {\"A}nderungsbeschluss
		\item Festlegung der weiteren Schritte
		\item Ergebnisse der Raumbesichtigung in der Dieskaustr. 20
    \end{itemize}

\section*{3) {\"U}berarbeitung der Satzung des Vereins nach Vorgaben des Amtsgerichts}
	Die Vorpr{\"u}fung des Amtsgerichtes hat ergeben, dass 2 Punkte in der
	Satzung geändert werden m{\"u}ssen. Auszug aus der betreffenden Mail:
\subsection*{Eingehende Mail}
	Betreff:     dezentrale e.V. - 43 AR 88/2017\\
	Datum:     Mon, 19 Jun 2017 06:37:54 +0000\\
	Von:     Ertel, Gitta - Justiz Sachsen, AG Leipzig\\\\
		in Beantwortung Ihrer E-Mail vom 15.06.2017 teile ich mit, dass die Satzung mit
		Ausnahme der schriftlichen Stimmabgabe inhaltlich nicht zu beanstanden ist.\\\\
		\S\ 11.1 verlagert konkrete Bestimmungen hier{\"u}ber in die Gesch{\"a}ftsordnung.
		Das ist nicht zul{\"a}ssig. Die Satzung selbst muss die schriftliche/elektronische
		Stimmabgabe klar regeln.\\\\
		F{\"u}r \S\ 7.8 gilt, dass die \qs{Kann}-Bestimmung {\"u}ber die Einladung durch
		eine \qs{Ist}-Reglung zu ersetzen ist. So wird eine eindeutige Festlegung geschaffen,
		die nicht hinterfragt werden muss, denn die Form der Einladung ist durch die Satzung
		eindeutig zu regeln (z.B. Der Vorstand stellt die in Textform abgefassten Einladungen
		gem. \S\ 11 der Satzung zu. Er muss jedoch eine Kopie auf dem Postweg...).\\\\
		Dank der im \S\ 9.3 der Satzung bestimmten Erm{\"a}chtigung ist die {\"A}nderung
		durch Vorstandsbeschluss gestattet.
\subsection*{{\"A}nderungsbeschluss}
	\begin{enumerate}
		\item Satzung \S\ 7.8 wird ge{\"a}ndert.\\
			  Alter \S\ 7.8:\\
			  \qs{Der Vorstand kann die Einladungen auf schriftlichem Weg
			  gem{\"a}{\ss} \S \ 11 zustellen, muss jedoch eine Kopie auf dem Postweg
			  zustellen, falls das Mitglied den Wunsch dazu schriftlich gem{\"a}{\ss}
			  \S \ 11 angemeldet hat.}\\
			  Neuer \S\ 7.8:\\
			  \qs{Der Vorstand stellt die Einladungen auf schriftlichem Weg
			  gem{\"a}{\ss} \S \ 11 zu, muss jedoch eine Kopie auf dem Postweg
			  zustellen, falls das Mitglied den Wunsch dazu schriftlich gem{\"a}{\ss}
			  \S \ 11 angemeldet hat.}
		\item Satzung \S\ 11.1 wird ge{\"a}ndert. In der {\"U}berschrift wird \qs{Schriftform} durch \qs{Schriftliche Kommunikation} ersetzt.\\
			  Alter \S\ 11.1:\\
			  \qs{Schriftliche Erkl{\"a}rungen im Sinne dieser Satzung
			  k{\"o}nnen auch elektronische Dokumente sein. Die Gesch{\"a}ftsordnung
			  bestimmt Anforderungen, Zustellwege und Zuordnung derartiger Dokumente.}\\
			  Neuer \S\ 11.1:\\
			  \qs{Schriftliche Erkl{\"a}rungen im Sinne dieser Satzung
			  sind handschriftlich unterschriebene Dokumente in Papierform sowie
			  mit PGP oder S/MIME signierte elektronische Dokumente.}
		\item Satzung \S\ 11.2 wird  ge{\"a}ndert.\\
			  Alter \S\ 11.2:\\
			  \qs{Zu Mitgliederversammlungen werden elektronisch nach Abs. 1
			  oder postalisch zugestellte Stimmen von Mitgliedern wie Stimmen von anwesenden
			  Mitgliedern gez{\"a}hlt.}\\
			  Neuer \S\ 11.2:\\
			  \qs{Zu Mitgliederversammlungen werden elektronisch nach Abs. 1,
			  fernm{\"u}ndlich oder postalisch zugestellte Stimmen von Mitgliedern wie Stimmen
			  von anwesenden Mitgliedern gez{\"a}hlt.}
		\item Da die Satzung \S\ 11.1 keine Referenz mehr auf die Gesch{\"a}ftsordnung enth{\"a}lt,
			  wird der entsprechende \S\ 8 dort gestrichen. \S\ 9 wird zu \S\ 8.\\
	\end{enumerate}
\section*{4) Abstimmung {\"u}ber den {\"A}nderungsbeschluss}
	\begin{enumerate}
		\item Die {\"A}nderung von \S\ 7.8 wird einstimmig von der Vorstandssitzung beschlossen.\\
		\item Die {\"A}nderung von \S\ 11.1 wird einstimmig von der Vorstandssitzung beschlossen.\\
		\item Die {\"A}nderung von \S\ 11.2 wird einstimmig von der Vorstandssitzung beschlossen.\\
		\item Die Streichung von \S\ 8 in der Gesch{\"a}ftsordnung wird einstimmig von der Vorstandssitzung beschlossen.\\
	\end{enumerate}

\section*{5) Festlegung der weiteren Schritte}
    Die weiteren Schritte werden wie folgt abgearbeitet:
	\begin{itemize}
		\item Die ge{\"a}nderte Satzung wird am 22.06.2017 beim Notar eingereicht und alle fehlenden Daten dort erg{\"a}nzt.
		\item Es wird in einem oder mehreren Terminen beim Notar beglaubigte Unterschriften geleistet.
		\item Der Notar beantragt die Eintragung des Vereins beim Amtsgericht.
		\item Nach erfolgter Eintragung als e.V. muss der Verein noch beim Finanzamt angemeldet werden.
	\end{itemize}

\section*{6) Ergebnisse der Raumbesichtigung in der Dieskaustr. 20}
    Am 21.06.2017 fand die Besichtigung o.g. R{\"a}ume statt.\\
	(allg. Diskussion und Pr{\"a}sentation der Besichtigungsergebnisse)\\
	
\section*{7) Ende der Sitzung}
    Mit ihren Unterschriften best{\"a}tigt der Schriftf{\"u}hrer die Inhalte dieses Protokolls.
\\
\\
\\
\\
\\
\\
Leipzig, den \eventdate \ \ \ Schriftf{\"u}hrer: \schriftfuehrer
\end{document}

