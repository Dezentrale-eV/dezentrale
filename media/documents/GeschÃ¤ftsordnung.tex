\documentclass[10pt,a4paper]{scrartcl}
\usepackage{eurosym}
\usepackage[margin=3cm]{geometry}
\usepackage[utf8]{inputenc}
\newcommand{\qt}[1]{\glq\emph{#1}\grq}
\newcommand{\qs}[1]{"#1"}
\newcommand{\name}{dezentrale}
\newcommand{\revision}{$4$}
\newcommand{\revdate}{$2018-04-20$}
\setlength{\parskip}{6pt}
\setlength{\parindent}{0pt}
\usepackage{enumerate}
\usepackage{color}
\usepackage{svg}
\pagestyle{plain}
\usepackage{palatino}
\usepackage[bookmarks,bookmarksopen=true,bookmarksnumbered=true,colorlinks,linkcolor=black,urlcolor=blue]{hyperref}
\begin{document}

{\setlength\tabcolsep{0 pt}
\begin{tabular}{p{10.6cm}p{4.7cm}} {\LARGE Gesch{\"a}ftsordnung des \name\ e.V.} &
	\begin{tabular}{@{}c@{}}
	\includesvg[width=4.5cm]{dz_logo_vektor_alpha}
	\end{tabular}
\end{tabular}
}
Fassung vom \revdate\ (Revision \revision)

%
% Paragraph 1 ====================================================
%
\subsection*{\S \ 1 Mitgliedsbeitr{\"a}ge und Mitgliedsarten}
% 5.3
\begin{enumerate}
\item Der Verein erhebt gem{\"a}{\ss} \S \ 5 seiner Satzung Mitgliedsbeitr{\"a}ge wie folgt:
	\begin{itemize}
    \item 16,00 \euro / Monat f{\"u}r erm{\"a}{\ss}igte Mitgliedschaft
    \item 32,00 \euro / Monat f{\"u}r normale Mitgliedschaft
	\item 42,00 \euro / Monat, 64,00 \euro / Monat, 128,00 \euro / Monat oder 256,00 \euro / Monat f{\"u}r Nerdmitglieder
	\end{itemize}
    

    \item Mitglieder sind verpflichtet, den Mitgliedsbeitrag bis zum dritten Werktag des Monats
    auf das Konto des Vereines zu {\"u}berweisen. Die Mitgliedschaft endet bei einem R{\"u}ckstand
    von zwei Monatsbeitr{\"a}gen automatisch. Ausnahmen k{\"o}nnen vom Vorstand beschlossen werden.
    Bei automatischer Beendigung der Mitgliedschaft wird das Mitglied schriftlich {\"u}ber den
    Vorgang informiert.

    \item Vorrausetzung f{\"u}r die erm{\"a}{\ss}igte Mitgliedschaft erf{\"u}llen alle Personen, welche Sch{\"u}ler, Studenten,
    Rentner, Gefl{\"u}chtete, oder Empf{\"a}nger von staatlicher Hilfe sind, und dies j{\"a}hrlich nachweisen.

\item Die Nerdmitgliedschaft steht jedem Mitglied zur Wahl, das den Verein st{\"a}rker
	finanziell unterst{\"u}tzen m{\"o}chte. Mit einer solchen
	Nerdmitgliedschaft sind keinerlei Privilegien oder Stimmvorteile gegen{\"u}ber
	den anderen beiden Mitgliedschaften verbunden.
    
    \item Ein Mitgliedschaftsanw{\"a}rter hat die Pflicht, sich einem Vorstandsmitglied oder zur 
    Mitgliederversammlung pers{\"o}nlich vorzustellen, um die Mitgliedschaft wirksam zu machen.
    \item Unabh{\"a}ngig vom Mitgliedsbeitrag wird zwischen regul{\"a}ren Mitgliedern und F{\"o}r\-der\-mit\-gliedern
    unterschieden. Bei Eintritt in den Verein ist jedes Mitglied ein regul{\"a}res Mitglied.
    Nimmt ein regul{\"a}res Mitglied an zwei aufeinanderfolgenden Mitgliederversammlungen nicht
    teil, wird es automatisch zum F{\"o}rdermitglied. Bei Teilnahme an einer Mitgliederversammlung
    kann das Mitglied auf eigenen Wunsch wieder zur regul{\"a}ren Mitgliedschaft wechseln.
    F{\"o}rdermitglieder haben kein Stimmrecht in der Mitgliederversammlung.

    
\item {\"A}nderungen bez{\"u}glich der Mitgliedschaft (Mitgliedbeitrag, Mitgliedsart oder Beendigung der Mitgliedschaft)
	sind dem Vorstand schriftlich mitzuteilen.
\item	Mit Annahme des Mitgliedsantrages und Ausgabe der
	Mitgliedsnummer ist ein Mitgliedschaftsanw{\"a}rter ein
	vorl{\"a}ufiges Mitglied. Nach einer Frist von sieben
	Tagen und dem Eingang des Mitgliedsbeitrages wird das
	vorl{\"a}ufige Mitglied zum regul{\"a}ren Mitglied.
\end{enumerate}
%
% Paragraph 2 ====================================================
%
\subsection*{\S \ 2 Beendigung der Mitgliedschaft}
% 3.5
\begin{enumerate}
\item Eine Beendigung der Mitgliedschaft ist mit sofortiger Wirkung m{\"o}glich.
\item Die Beitragszahlungspflicht endet zum entsprechenden Monatsende.
\end{enumerate}
%
% Paragraph 3 ====================================================
%
\subsection*{\S \ 3 Einschr{\"a}nkungen der Verf{\"u}gungsberechtigung des Vorstands}
% 8.2
\begin{enumerate}
\item Vorstandsmitglieder, die den Verein alleine nach au{\ss}en vertreten
    d{\"u}rfen, sind bei Rechtsgesch{\"a}ften bis zu einem Betrag von 500 EUR
    verf{\"u}gungsberechtigt. {\"U}ber einen Betrag von bis zu 5000 EUR muss der
    Vorstand abstimmen. Bei h{\"o}heren Betr{\"a}gen ist ein Beschluss durch die
    Mitgliederversammlung n{\"o}tig.
\item Bei fortlaufenden Vertr{\"a}gen wird die erwartete Summe
    {\"u}ber 6 Monate analog betrachtet.
\end{enumerate}
%
% Paragraph 4 ====================================================
%
\subsection*{\S \ 4 Mitgliederversammlung}
% 7.4.
\begin{enumerate}
\item Vor Feststellung der Beschlussf{\"a}higkeit der Mitgliederversammlung kann ein F{\"o}rdermitglied zur regul{\"a}ren Mitgliedschaft wechseln und damit an allen Beschl{\"u}ssen der Versammlung teilnehmen.
\item Die Untergrenze f{\"u}r die Beschlussf{\"a}higkeit gem{\"a}{\ss} Satzung \S \ 7 betr{\"a}gt 51\% der regul{\"a}ren Mitglieder.
\item Die teilnehmenden regul{\"a}ren Mitglieder sind im Versammlungsprotokoll zu protokollieren.
\item G{\"a}ste sind zur Mitgliederversammlung zugelassen.
\item Als Wahlverfahren f{\"u}r die Vorstandswahl wird standardm{\"a}{\ss}ig das \qs{Instant-Runoff-Verfahren} eingesetzt.
\end{enumerate}
%
% Paragraph 5 ====================================================
%
\subsection*{\S \ 5 Grunds{\"a}tze der Verm{\"o}gensverwaltung des Vereins}
\begin{enumerate}
\item Der Vorstand hat Sorge zu tragen, dass das Gesamtverm{\"o}gen des Vereins nicht negativ wird.
\item Es ist anzustreben, dass die liquiden Mittel mindestens 1,5 Anteile der monatlichen durchschnittlichen Fixkosten als Reserve betragen.
\item Sollte die Mindestmenge an Liquidit{\"a}tsreserve angegriffen werden, so sind dar{\"u}ber umgehend alle Mitglieder zu informieren.
\end{enumerate}

%
% Paragraph 6 ====================================================
%
\subsection*{\S \ 6 Aufgaben des Schatzmeisters}
\begin{enumerate}
\item Der Schatzmeister hat auf eine sparsame und wirtschaftliche
	Haushaltsf{\"u}hrung hinzuwirken
\item Der Schatzmeister legt nach Eintragung des Vereines in das Vereinsregister
	ein Konto auf den Namen des Vereines an und verwaltet dort das
	Vereinsverm{\"o}gen.
\item Der Schatzmeister informiert die Vereinsmitglieder mindestens
    j{\"a}hrlich sowie innerhalb von acht Wochen nach gr{\"o}{\ss}eren
	Veranstaltungen, bei denen der Verein als Veranstalter oder
	Mitveranstalter auftritt, {\"u}ber den Kassenstand. Einnahmen und
	Ausgaben {\"u}ber 100 EUR sind dabei einzeln aufzulisten.
\item Als Vorstandsmitglied hat der Schatzmeister die Einbringung der
	Mitgliedsbeitr{\"a}ge und anderer Einnahmen zu organisieren. Dabei
	genie{\ss}t er die volle Unterst{\"u}tzung des Vorstands.
\item Der Schatzmeister kann seine Aufgaben in eigenem Ermessen delegieren.
\item F{\"u}r laufende Einnahmen und Ausgaben kann der Schatzmeister eine
    Bargeldkasse f{\"u}hren. {\"U}ber\-sch{\"u}s\-sige Bargeldsummen werden von ihm
	regelm{\"a}{\ss}ig auf dem Vereinskonto abgelegt.
\item F{\"u}r Bareing{\"a}nge stellt der Schatzmeister eine formgerechte Quittung
	in doppelter Ausfertigung aus, davon eine f{\"u}r den Einzahler.
\item Der Schatzmeister legt ein geeignetes Verm{\"o}gensregister an, das
	nach den Regeln der einfachen Buchf{\"u}hrung zu f{\"u}hren ist und aus
	folgenden Teilen besteht:
	\begin{itemize}
	\item Kassenbuch f{\"u}r die Bargeldkasse
	\item Hauptbuch f{\"u}r das Vereinskonto
	\item Inventarliste f{\"u}r Verm{\"o}gensgegenst{\"a}nde 
	\end{itemize}
\item Jede einzelne Ausgabe muss belegt werden. Jeder Beleg muss von
	dem Vereinsmitglied, das die Ausgabe get{\"a}tigt hat, umgehend
	beim Schatzmeister eingereicht werden.
\item Der Schatzmeister f{\"u}hrt die Liste der Vereinsmitglieder.
	Periodisch werden von ihm die sich ergebenden Ver{\"a}nderungen
	durch Zug{\"a}nge und Abg{\"a}nge den Vereinsmitgliedern mitgeteilt.
\item Bei Wechsel des Schatzmeisters ist durch ihn eine {\"U}bersicht zu erstellen.
\end{enumerate}
%
% Paragraph 7 ====================================================
%
\subsection*{\S \ 7 Erstattung der Auslagen des Vorstands}
% 8.8
\begin{enumerate}
\item Auslagen des Vorstandes zur Verfolgung der Vereinszwecke werden
	in voller H{\"o}he erstattet. Auf Beschluss der Mitgliederversammlung
	muss der Vorstand in einer Stellungnahme Zweck- und Verh{\"a}ltnis\-m{\"a}{\ss}igkeit
	der Ausgaben nachweisen.
\end{enumerate}
%
% Paragraph 8 ====================================================
%
\subsection*{\S \ 8 Infrastruktur}
%
\begin{enumerate}
\item Die Infrastruktur des Vereins muss hinreichend dokumentiert werden.
\end{enumerate}

\end{document}

